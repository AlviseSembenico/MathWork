\documentclass{article}
\usepackage[utf8]{inputenc}
\usepackage{listings}
\usepackage{amsfonts} % For other mathematical fonts if needed
\usepackage{amsmath}
\usepackage{graphicx}
\usepackage{soul}
\usepackage{hyperref}
\usepackage{mathtools}
\usepackage{amsthm}
\usepackage{wasysym}
\usepackage{bbm} % Use bbm for blackboard bold indicator function
\usepackage{xcolor}

\newcommand{\comment}[1]{%
  \text{\phantom{(#1)}} \tag{#1}
}
\newcommand{\Cov}{\mathrm{Cov}}
\newcommand\myeq{\mathrel{\overset{\makebox[0pt]{\mbox{\normalfont\tiny\sffamily def}}}{=}}}

\title{Portfolio Theory\\
Homework 1}
\author{
  Sembenico, Alvise\\
  \texttt{12380288} \\  
}
\date{}

\begin{document}
\maketitle

\section{1.1}
As in this exercise we are looking at several stochastic processes and have to classify as predictable or just adapted, we will first show that the indicator function is adapted/predictable if the argument is adapted/predictable.
This is true almost just by definition as the preimage is the following.

\begin{equation}
  \mathbbm{1}_{A}^{-1}(x) = \begin{cases}
    A \text{ if }x=1                  \\
    \Omega \setminus A \text{ if }x=0 \\
  \end{cases}
\end{equation}
Where \(A \in  \mathcal{F}\) so does \(\Omega \setminus A = A^C\).
With this in mind, we can proceed on considering the following processes.
\begin{itemize}
  \item $\varphi_t = \mathbbm{1}_{\{ S_t^{(1)} > S_{t-1}^{(1)} \}};$
        \subitem \(\phi _t\) is merely adapted as \(S_t\) is just adapted.

  \item $\varphi_1 = 1$ and $\varphi_t = \mathbbm{1}_{\{ S_{t-1}^{(1)} > S_{t-2}^{(1)} \}}$ for $t \geq 2$;
        \subitem \(\phi_t\) is predictable as both process are \(\mathcal{F}_t\) measurable, thus, \(\phi _{t-1}\) is \(\mathcal{F}_t\) measurable.

  \item $\varphi_t = \mathbbm{1}_A \cdot \mathbbm{1}_{\{ t > t_0 \}},$ where $t_0 \in \{ 0, \dots, T \}$ and $A \in \mathcal{F}_{t_0};$
        \subitem We can see that \( \mathbbm{1}_A\) and \( \mathbbm{1}_{t>t_0}\) are both deterministic functions, moreover, given that \(A \in  \mathcal{F}_0\), we have that for every \(t\geq 1\), \(\phi _t\) is \(F_{t+1}\) measurable, therefore the process is predictable.

  \item $\varphi_t = \mathbbm{1}_{\{ S_t^{(1)} > S_0^{(1)} \}};$
        \subitem Again by looking at the argument of the argument of the indicator function, we see that \(S_t\) is merely adapted. It follows that \(\phi _t \) is also just adapted\dots

  \item $\varphi_1 = 1$ and $\varphi_t = 2 \varphi_{t-1} \mathbbm{1}_{\{ S_{t-1}^{(1)} < S_0^{(1)} \}}$ for $t \geq 2.$
        \subitem We can see that the argument of the indicator function is again predictable. We have to be careful about the \(\phi _{t-1}\) component. However, using an induction argument, we can see that each \(\phi _{t-1}\) is \(\mathcal{F}_{t}\) measurable, making it predictable. It follows that \(\phi _t \) is predictable as well.
\end{itemize}


\section{1.2}
\begin{proof}
  We will prove the statement with a series of double direction implications.


  A strategy is self financing if and only if
  \[
    W_t(\phi )= W_0(\phi ) + G_t(\phi ) =  W_0(\phi ) + (\phi \cdot  X)_t
  \]
  For every \(t\).
  It follows that
  \begin{align*}
    \phi _t^T S_t                                       & =\phi _0 ^TS_0 + \sum _i^t \phi_{i}^T  (S_i-S_{i-1}) \\
    \sum_{i}^t \phi _i ^T S_{i} - \phi _{i-1}^T S_{i-1} & = \sum _i^t\phi_i^T (S_i-S_{i-1})                    \\
    \sum_{i}^t(\phi _i^T + \phi_{i-1}^T )S_{i-1}        & =0
  \end{align*}
  For every \(t=0,\dots ,T\). As it has to be true for all the \(t\), by induction, we deduce that
  \[
    (\phi _t^T - \phi _{t-t}^T) S_{t-1}= 0.
  \]
  for all \(t=1,\dots T\). In other words, given that it's true for \(t=1\)
  and by the inductive step
  \begin{align*}
    \sum_{i}^2(\phi _i^T + \phi_{i-1}^T )S_{i-1} & =  (\phi _1^T + \phi_{0}^T )S_{0}+  (\phi _2^T + \phi_{1}^T )S_{1} \\
                                                 & = (\phi _2^T + \phi_{1}^T )S_{1} =0.
  \end{align*}
  By diving both sides of the equation, we get that
  \begin{equation}
    (\phi _t^T - \phi _{t-t}^T) \widetilde{S}_{t-1}= 0.
  \end{equation}
  Given that this is true for every \(t\), again by induction argument, we conclude that
  \begin{equation}
    \widetilde{W}_t(\phi ) = \widetilde{W}_0(\phi )+(\phi \cdot \widetilde{X} )_t.
  \end{equation}
  For every \(t=0,\dots ,1\).
\end{proof}

\section{1.3}
\begin{proof}
  \begin{align*}
    \Delta W_t(\phi )                                         & = \phi _t^T \Delta X_t      \\
    \iff               \phi _t ^T S_t - \phi _{t-1}^T S_{t-1} & = \phi_t S_t - \phi S_{t-1} \\
    \iff \phi _{t-1}^T S_{t-1}                                & = \phi _t^T S_{t-1}         \\
    \iff  \phi _t^T S_t                                       & = \phi _{t+1}^{T}S_t
  \end{align*}
  where the last statement holds true as the previous ones are true for every \(t =0,\dots ,T\).
\end{proof}

\section{1.4}
Let's start by noting that the reciprocal absolute continuity of \(\mathcal{P}, \mathbb{Q}\) implies that the Radon Nikodym is well defined.
Moreover, we can define
\begin{equation}
  Z_\infty := \frac{dQ}{dP} \bigg|_{\mathcal{F}_\infty}.
\end{equation}
Furthermore, we define
\begin{equation}
  Z_t = \mathbb{E} \left[ Z_\infty  |\mathcal{F}_t \right].
\end{equation}
We now show that \(Z_t\) is indeed a martingale. We know that it's squared integrable, thus \(\mathbb{E} \left[ |Z_t| \right] <\infty  \forall t>0 \). So we focus on the martingality property.
For \(s\leq t\);
\begin{align*}
  \mathbb{E} \left[ Z_t|\mathcal{F}_s \right] & = \mathbb{E} \left[ \mathbb{E} \left[ Z_\infty |F_t \right] |\mathcal{F}_s \right] \\
                                              & = \mathbb{E} \left[ Z_\infty |\mathcal{F}_{s \wedge t} \right]                     \\
                                              & = Z_s
\end{align*}

Let's now show that
\begin{equation}
  Z_t = \frac{dQ}{dP}\bigg|_{\mathcal{F}_t}
\end{equation}
Take \(A \in  F_t\), then
\begin{equation}
  Q(A) = \mathbb{E}_P \left[ \mathbbm{1}_A \frac{dQ}{dP}\bigg|_{\mathcal{F}_t} \right]
\end{equation}
But also, for \(A \in  \mathcal{F}_t \subset \mathcal{F}_\infty  \)
\begin{align}
  Q(A) & = \mathbb{E}_P \left[ \mathbbm{1}_A Z_\infty  \right]                                           \\
       & = \mathbb{E}_P \left[ \mathbb{E}_P \left[ \mathbbm{1}_A Z_\infty |\mathcal{F}_t \right] \right] \\
       & = \mathbb{E}_P \left[ \mathbbm{1}_A  Z_t\right]
\end{align}
From both equalities we conclude that
\begin{equation}
  Z_t = \frac{dQ}{dP}\bigg|_{\mathcal{F}_t}
\end{equation}
% Let's set
% \begin{equation}
%   \hat{Z} \coloneqq \frac{dQ}{dP}\bigg|_{\mathcal{F}_t}
% \end{equation}
% Then, by Lemma 9.2 of Stochastic integration
% \begin{equation}
%   \hat{Z}= \frac{\mathbb{E}_P \left[ X Z_\infty |\mathcal{F}_t \right]}{\mathbb{E}_{Q}  \left[ X|\mathcal{F}_t \right]} = Z_t \frac{\mathbb{E}_p \left[ X|\mathcal{F}_t \right]}{\mathbb{E}_Q \left[X|\mathcal{F}_t  \right]}
% \end{equation}



\end{document}
