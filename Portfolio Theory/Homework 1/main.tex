\documentclass{article}
\usepackage[utf8]{inputenc}
\usepackage{listings}
\usepackage{amsfonts} % For other mathematical fonts if needed
\usepackage{amsmath}
\usepackage{graphicx}
\usepackage{soul}
\usepackage{hyperref}
\usepackage{mathtools}
\usepackage{amsthm}
\usepackage{wasysym}
\usepackage{bbm} % Use bbm for blackboard bold indicator function
\usepackage{xcolor}

\newcommand{\comment}[1]{%
  \text{\phantom{(#1)}} \tag{#1}
}
\newcommand{\Cov}{\mathrm{Cov}}
\newcommand\myeq{\mathrel{\overset{\makebox[0pt]{\mbox{\normalfont\tiny\sffamily def}}}{=}}}

\title{Portfolio Theory\\
Homework 1}
\author{
  Sembenico, Alvise\\
  \texttt{12380288} \\  
}
\date{}

\begin{document}
\maketitle

\section{1.1}
As in this exercise we are looking at several stochastic processes and have to classify as predictable or just adapted, we will first show that the indicator function is adapted/predictable if the argument is adapted/predictable.
This is true almost just by definition as the preimage is the following.

\begin{equation}
  \mathbbm{1}_{A}^{-1}(x) = \begin{cases}
    A \text{ if }x=1                  \\
    \Omega \setminus A \text{ if }x=0 \\
  \end{cases}
\end{equation}
Where \(A \in  \mathcal{F}\) so does \(\Omega \setminus A = A^C\).
With this in mind, we can proceed on considering the following processes.
\begin{itemize}
  \item $\varphi_t = \mathbbm{1}_{\{ S_t^{(1)} > S_{t-1}^{(1)} \}};$
        \subitem \(\phi _t\) is merely adapted as \(S_t\) is just adapted.

  \item $\varphi_1 = 1$ and $\varphi_t = \mathbbm{1}_{\{ S_{t-1}^{(1)} > S_{t-2}^{(1)} \}}$ for $t \geq 2$;
        \subitem \(\phi_t\) is predictable as both process are \(\mathcal{F}_t\) measurable, thus, \(\phi _{t-1}\) is \(\mathcal{F}_t\) measurable.

  \item $\varphi_t = \mathbbm{1}_A \cdot \mathbbm{1}_{\{ t > t_0 \}},$ where $t_0 \in \{ 0, \dots, T \}$ and $A \in \mathcal{F}_{t_0};$
        \subitem We can see that \( \mathbbm{1}_A\) and \( \mathbbm{1}_{t>t_0}\) are both deterministic functions, moreover, given that \(A \in  \mathcal{F}_0\), we have that for every \(t\geq 1\), \(\phi _t\) is \(F_{t+1}\) measurable, therefore the process is predictable.

  \item $\varphi_t = \mathbbm{1}_{\{ S_t^{(1)} > S_0^{(1)} \}};$
        \subitem Again by looking at the argument of the argument of the indicator function, we see that \(S_t\) is merely adapted. It follows that \(\phi _t \) is also just adapted\dots

  \item $\varphi_1 = 1$ and $\varphi_t = 2 \varphi_{t-1} \mathbbm{1}_{\{ S_{t-1}^{(1)} < S_0^{(1)} \}}$ for $t \geq 2.$
        \subitem We can see that the argument of the indicator function is again predictable. We have to be careful about the \(\phi _{t-1}\) component. However, using an induction argument, we can see that each \(\phi _{t-1}\) is \(\mathcal{F}_{t}\) measurable, making it predictable. It follows that \(\phi _t \) is predictable as well.
\end{itemize}


\section{}


\end{document}
