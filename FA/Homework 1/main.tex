\documentclass[a4paper,12pt]{article} % This defines the style of your paper

\usepackage[top = 2.5cm, bottom = 2.5cm, left = 2.5cm, right = 2.5cm]{geometry} 

% fonts
\usepackage[T1]{fontenc}
\usepackage[utf8]{inputenc}
\usepackage{xcolor}
\usepackage{amsmath}
\usepackage{amsfonts}
\usepackage{amssymb}
\usepackage{enumitem}
\usepackage{amsthm}
\usepackage{mathtools}
  
% The following two packages - multirow and booktabs - are needed to create nice looking tables.
\usepackage{multirow} % Multirow is for tables with multiple rows within one cell.
\usepackage{booktabs} % For even nicer tables.

% The default setting of LaTeX is to indent new paragraphs. This is useful for articles. But not really nice for homework problem sets. The following command sets the indent to 0.
\usepackage{setspace}
\setlength{\parindent}{0in}

% Package to place figures where you want them.
\usepackage{float}

% The fancyhdr package let's us create nice headers.
\usepackage{fancyhdr} 

\newtheorem{proposition}{Proposition}

%%%%%%%%%%%%%%%%%%%%%%%%%%%%%%%%%%%%%%%%%%%%%%%%
% Header (and Footer)
%%%%%%%%%%%%%%%%%%%%%%%%%%%%%%%%%%%%%%%%%%%%%%%%

\pagestyle{fancy} % With this command we can customize the header style.

\fancyhf{} % This makes sure we do not have other information in our header or footer.

\lhead{\footnotesize Functional Analysis}% \lhead puts text in the top left corner. \footnotesize sets our font to a smaller size.

%\rhead works just like \lhead (you can also use \chead)
\rhead{\footnotesize Alvise Sembenico}%, Lastname 2 (\& Lastname 3)} %<---- Fill in your lastnames.

% Similar commands work for the footer (\lfoot, \cfoot and \rfoot).
% We want to put our page number in the center.
\cfoot{\footnotesize \thepage} 


%%%%%%%%%%%%%%%%%%%%%%%%%%%%%%%%%%%%%%%%%%%%%%%%
% Custom commands
\newcommand{\comment}[1]{%
  \text{\phantom{(#1)}} \tag{#1}
}
%%%%%%%%%%%%%%%%%%%%%%%%%%%%%%%%%%%%%%%%%%%%%%%%

%%%%%%%%%%%%%%%%%%%%%%%%%%%%%%%%%%%%%%%%%%%%%%%%
% Document
%%%%%%%%%%%%%%%%%%%%%%%%%%%%%%%%%%%%%%%%%%%%%%%%
\begin{document}
\begin{center} % Everything within the center environment is centered.
    {\Large \bf Homework 1}
\end{center}

\vspace{0.4cm}

%%%%%%%%%%%%%%%%%%%%%%%%%%%%%%%%%%%%%%%%%%%%%%%%
% THE HOMEWORK
% Can be written right here or in the dedicated files
%%%%%%%%%%%%%%%%%%%%%%%%%%%%%%%%%%%%%%%%%%%%%%%%

\onehalfspacing
\section{Exercise 1}
\subsection{}
Let's define \(x_0\) as the limit of the net
\begin{equation}
    \sum_{a \in  A}a \to x_0.
\end{equation}
Since it converges, it means that there exists a set \(F_0 \subset A\) such that for all \(F > F_0\) the following holds
\begin{equation}
    \| \sum_{a \in  F}a - x_0  \| < \epsilon
\end{equation}
for a given \(\epsilon >0\). Since \(\epsilon \) is arbitrary, we can choose \(F^{\prime} _0> F_0\) such that for all \(F > F_0^{\prime} \)

\begin{equation}
    \| \sum_{a \in  F}a - x_0  \| < \frac{\epsilon}{|\alpha| }.
\end{equation}
Then, by properties of the norm, we obtain
\begin{align*}
    \vert \alpha \vert \| \sum_{a \in  F}a - x_0  \| & < \frac{\epsilon}{|\alpha| }\alpha = \epsilon \\
    \| \alpha \sum_{a \in  F}a - \alpha x_0  \|      & < \epsilon                                    \\
    \|  \sum_{a \in  F}\alpha a - \alpha x_0  \|     & < \epsilon.
\end{align*}
Where in the last step we used the fact that \(F\) is finite.
This proves that \(\alpha \sum_{a \in  A} a\) converges to \(\alpha x_0 = \alpha \sum_{a \in  A} a \).

\subsection{}
The hypothesis that \(\sum_{a \in  A}a  \) and \(\sum_{b \in  B}b  \) implies that there exists an \(F_0^a\) and \(F_0^b\) such that for every \(F^a>F_0^a\) and \(F^b>F_0^b\) the following holds
\begin{equation}
    \| \sum_{a \in  F^a}a - \sum_{a \in  A}a   \| < \frac{\epsilon}{2} \quad \quad \| \sum_{b \in  F^b}b - \sum_{b \in  B}b   \|< \frac{\epsilon}{2}.
\end{equation}
Denote \(F_0 = F_0^a \cup F_0^b\), it follows that for every \(F > F_0\)
\begin{align*}
    \| \sum_{x \in  F}x - \sum_{a\in A}a - \sum_{b \in  B }b    \| & = \| \sum_{x \in  F \cap A} + \sum_{x \in F \cap B}x - \sum_{a \in  A}a - \sum_{b \in  B}b     \|                                                             \\
                                                                   & \leq \| \sum_{x \in  F \cap A}x - \sum_{a \in  A}a   \|+\| \sum_{x \in F \cap B}x - \sum_{b \in  B }b \|< \frac{\epsilon}{2} + \frac{\epsilon}{2} = \epsilon.
\end{align*}
Because \(\epsilon\) was arbitrary, we conclude that \(\sum_{x \in A \cup B}\to \sum_{a \in  A} a + \sum_{b \in  B}b   \).

\subsection{}
Let's start by proving the following proposition:

\begin{proposition}
    Any converging \(\sum_{a \in  A}a \to x\)  of positive numbers has at most a countable number of non-zero elements. That is, \(\sum_{a \in A}x = \sum_{i \in  \mathbb{N}}a_i \), meaning that they converge to the same value \(x\).
\end{proposition}
\begin{proof}
    Say that the net converges to \(M\), i.e. \(\sum_{a \in  A} a = M <\infty  \) where for every \(a \in A, a>0\). Consider now the sets \(S_n = \{ a \in  A | a> \frac{1}{n} \} \), then
    \begin{align*}
        M \geq \sum_{a \in  S_n}a \geq  \sum_{a \in  S_n}\frac{1}{n}= \frac{N}{n}. \\
    \end{align*}
    As \(M < \infty \)  so is \(N\) which is the cardinality of the set \(S_n\).
    It follows that
    \begin{equation}
        \# \{ a \in  A | a>0 \}= \# S = \# \bigcup_{n=\mathbb{N}}^{\infty} S_n
    \end{equation}
    We conclude that \(A\) has at most countable number of non-zero elements as a countable union of finite sets.
\end{proof}


Let's now prove the \((\implies) \) direction.
Given the previously proven statement, we can rewrite the net as a countable sum and thus define a corresponding sequence \(x_n = \sum_{i=0}^n a_{i}  \) where w.log. we associated every non zero element \(a\)  of \(A\) to an index \(i\) so that \(a_i = a\).
From standard analysis we obtain that every converging increasing sequence is bounded from above, i.e. there exists \(N \in  R\) so that \(x_n < N\) for every \(n\).
It follows that for every finite \(F \subset I\)
\begin{equation}
    \sum_{a \in F} a \leq \sum_{i \in  \mathbb{N}}a_i \leq N.
\end{equation}


We now prove the opposite implication \((\impliedby)\). Assume that \(\sup \left\{ \sum_{a \in F} a : F \in \mathcal{F} \right\} = x_0\). We proceed now by contradiction, suppose that \(\sum_{a \in  A}a \to  x_0 +t \) for an arbitrary \(t>0\).


Let's define now what it means for a net to increase. Given a set \(F_0\), it holds that for every \(F > F_0\) we have that \(\sum_{a\in  F}a \geq  \sum_{a \in  F_0}a  \).

By defining net convergence for standard set inclusion as order, we obtain that the net \(\sum_{a\in  A}a \) is increasing.
It follows that for every \(F\), \(\sum_{a \in  F} \leq  x_0 +t \). Moreover, since \(\sum_{a \in  A} a \to x_{0}+t  \), there exists and \(F_t\)  such that \(\forall F > F_t\) we have \(\| \sum_{a \in  F} a -x_{0}-t   \| < \frac{t}{2} \). It follows
\begin{align*}
    \| \sum_{a \in  F} a -x_0 -t  \| & \leq 0                                   \\
    \| \sum_{a \in  F} a -x_0 -t  \| & =  \sum_{a \in  F} a -x_0 -t<\frac{t}{2} \\
    \sum_{a \in  F}a > x_0 +\frac{t}{2}
\end{align*}
Which is a contradiction. As this holds for any arbitrary \(t\), \(\sum_{a\in A}a \leq x_0 \).


Let's conclude the proof by showing that
\begin{equation}
    \sup \left\{ \sum_{a \in F} a : F \in \mathcal{F} \right\} = \sum_{a \in  A} a = x_0.
\end{equation}
Take an arbitrary \(x<x_0\). By contradiction, assume that \(\sup \left\{ \sum_{a \in F} a : F \in \mathcal{F} \right\} = x\).
Again, by convergence of the net, there exists a \(F_0\) such that \(\forall F>F_0\) we have \(    \| \sum_{a \in F}a -x_0  \|  <|x-x_0|\). As both quantities are negative, it follows that
\(    \sum_{a \in  F} a- x_0 > x-x_0\) which leads to the contradiction \(\sum_{a \in  F} a > x \).
Combining it with the previous point we get the desired equality.
% \textcolor{red}{To double check that this is all we need.}


\section{Exercise 2}
\subsection{}
Let's define \(H^{\prime}  = \bigvee \mathcal{F}\). By theorem 4.13 we have that \(\forall h \in  H^{\prime} \), \(h\) can be written as \(h = \sum_{e \in  F} \langle h,e \rangle e  \) as \(F\) is the basis for \(H^{\prime} \) again by theorem 4.13.

Moreover, for every \(x \in H\), we have that by definition \(P_F x \in  H^{\prime} \).
We define the operator \(Q\) acting on \(x\) as such
\begin{equation}
    Qx \coloneqq \sum_{e \in  F}\langle x,e \rangle e.
\end{equation}

For \(Q\) to be equal to \(P_F\), \(Qx \) has to be the unique elements in \(H^{\prime} \) such that \(x - Qx \perp H^{\prime} \).
We proceed by taking an orthogonal basis of \(H\) such that \(\mathcal{E} \subset B\), this basis is guaranteed to exist given proposition 4.2.
By theorem 4.13 again, by trivially noting that \(\bigvee H = H\), \(x\) can be represented as \(x=\sum_{e\in B}\langle x,e \rangle e  \).
It follows that
\begin{align*}
    x-Qx & = \sum_{e \in  B}\langle x,e \rangle e - \sum_{e \in  F}\langle x,e \rangle e                                                   \\
         & =  \sum_{e \in F}\langle x,e \rangle e+ \sum_{e \in  B\setminus F}\langle x,e \rangle e  - \sum_{e \in  F}\langle x,e \rangle e \\
         & = \sum_{e \in  B\setminus F}\langle x,e \rangle e
\end{align*}
We conclude that since B is orthogonal, it follows that \(x-Qx \perp H^{\prime} = \bigvee F\).


\subsection{}
By the previous point, we can write \(P_G x = \sum_{e \in  G}\langle x,e \rangle e \) for every \(G \subset \mathcal{G}\).
It follows that
\begin{align*}
    P_F P_G x & = P_F \left( \sum_{e \in  G}\langle x,e \rangle e  \right)
    \\
              & = \sum_{e \in  G} P_F \langle x,e \rangle e.                                                                \\
              & = \sum_{e \in  G} \sum_{e^{\prime}  \in F} \langle \langle e, x \rangle e, e^{\prime}  \rangle e^{\prime}   \\
              & = \sum_{e \in  G} \sum_{e^{\prime}  \in F}   \langle e, x \rangle \langle e, e^{\prime}  \rangle e^{\prime} \\
              & = \sum_{e \in  G \cap F}\langle e, x \rangle  e .
\end{align*}
Where we used the orthogonality of \(P_F\) and the fact that the set \(F,G\) are orthonormal.
% By using the orthogonality of the elements of \(F,G \subset \mathcal{E}\) we rewrite the above as follows.

% \begin{align*}
%     \sum_{e^{\prime}  \in F} \langle \sum_{e \in  G}\langle x,e \rangle e   ,e^{\prime} \rangle e' & = \sum_{ e \in  F \cap G}\langle \langle x,e \rangle e,e  \rangle \\
%                                                                                                    & = \sum_{e \in  F \cap G} \langle x,e \rangle  e = P_{F \cap G}x
% \end{align*}

% As this holds for arbitrary \(F, G\), proceeding the same fashion for \(P_G P_F\) completes the proof.

\subsection{}
Given an arbitrary finite family of disjoint orthonormal basis \((F_n)_{n \in \mathbb{N}}\).
\begin{align*}
    P_{\bigcup_{i=1}^{n} F_i } & = P_{\bigcup_{i=1}^{n-1}  \cup F_n} = P_{F_j \cup F_n} \\
\end{align*}
Where \(F_j \coloneqq  \bigcup_{i=1}^{n-1}F_i \). Thus, we just have to prove the above statement for two sets. This follows directly from the linearity of the sum, i.e.
\begin{align*}
    P_{F_i}x + P_{F_j}x & = \sum_{e \in F_i} \langle x,e \rangle e + \sum_{e \in  F_j} \langle x, e \rangle e = \sum_{e \in  F_i \cup F_j}\langle x,e \rangle e= P_{F_i \cup F_j}.
\end{align*}

For the convergence, we have that for a fixed \(\epsilon >0\), for all \(n>n_0\) the following holds true:
\begin{equation}
    \| P_{\bigcup_{i=1}^{\infty}F_i}x - \sum_{e \in  \bigcup_{i=1}^{n} F_i} \langle x,e \rangle  e  \|  = \| P_{\bigcup_{i=1}^{\infty}F_i}x - \sum_{ i=1}^n \sum_{e \in  F_i}\langle x,e \rangle e   \| < \epsilon.
\end{equation}
Equivalently, for the same \(\epsilon \), there exists a \(n_0^{\prime} \)  such that for all \(n>n_0\)
\begin{equation}
    \| \sum_{i} ^{\infty} P_{F_i}x - \sum_{i}^n P_{F_i}x  \| =      \| \sum_{i} ^{\infty} P_{F_i}x - \sum_{i}^n \sum_{e \in  F_i}\langle e,x \rangle x   \| < \epsilon
\end{equation}
By taking \(\overline{n} = \max (n_0, n_{0}^{\prime} )\) we get
that both limits converge to the same value. By uniqueness of the limit we have that
\begin{equation}
    P_{\bigcup_{i=1}^{\infty} F_i} x = \sum_{i=1}^{\infty} P_{F_i} x.
\end{equation}
% Where we used the exercise above for the last equation.
% By induction, we show that
% \begin{equation}
%     P_{\bigcup_{i=1}^{n} F_i }  = \sum_{i=1} ^n P_{F_i}
% \end{equation}
% We are now interest at the limit of those objects. Assume \(\lim_{n \to \infty} \sum_{i} ^n P_{F_i}x \to  x_0\) in the norm sense. Thus, there exists an \(n_0>0\) such that for every \(n>n_0\)
% \begin{equation}
%     \| \sum_{i=1} ^n P_{F_i}x - x_0 \| < \epsilon \\
% \end{equation}
% It follows that
% \begin{align*}
%     \| P_{} - x_0 \| < \epsilon \\ \\
% \end{align*}


\subsection{}
We prove by giving a counterexample that \(\sum_{i}^{\infty}  P_{F_i}x \) does not converge in the operator norm.
Let's define \(F_i= e_i\) where \(e_i \cap  e_j  = \varnothing \) and \(\| e_i \| = 1 \) for every \(i\).
Take now \(\epsilon =\frac{1}{2}\), then, for for every \(n>N\)
\begin{align*}
    \| \sum_{i=0}^{\infty} P_{F_i}x - \sum_{i=0}^n P_{F_i}x \| & = \| \sum_
    {i=n}^{\infty}  P_{F_i}x \| = 1 >\frac{1}{2}                            \\
\end{align*}
Therefore, for \(\epsilon =\frac{1}{2}\) there exists no \(n>0\) such that the the difference of the norm converges.

\section{Exercise 3}
\subsection{}
\subsubsection{Inner product}
We start by proving that \(H\) is indeed an inner product space given the definition of the inner product \(\langle x, y \rangle = \sum_{i \in I} \langle x(i), y(i) \rangle_i, \quad x, y \in H.
\)

The first property we prove is the conjugate symmetry.
For any \(x,y \in H\) and assume the net converges \(\langle x,y \rangle  \to  x_0\). Then, for a fixed \(\epsilon >0\), there exists an \(F_0\) such that for all \(F>F_0\)
\begin{align*}
    \| \overline{\langle x,y \rangle }_F - \overline{x_0}    \|                 & <\epsilon  \\
    \| \overline{\sum_{i \in  F}\langle x_i,y_i \rangle } - \overline{x_0}   \| & < \epsilon \\
    \| \sum_{i \in  F}\overline{\langle x_i,y_i \rangle }  -\overline{x_0}  \|  & < \epsilon \\
    \| \sum_{i \in  F}\langle y_i, x_i \rangle -\overline{ x_0}  \|             & <\epsilon.
\end{align*}
we conclude that \(\overline{\langle x,y \rangle } \to \overline{x_0}  \).

\subsubsection{Linearity in the first argument}
We approach the proof in the same fashion as in the previous part. Assume \(\langle x,y \rangle \) converges in \(H\) to \(x_0\) .
Then, for every \(F>F_0\)
\begin{align*}
    \| \langle x,y \rangle_F -x_0 \|                   & < \frac{\epsilon }{\lambda }                        \\
    |\lambda |\| \langle x,y \rangle_F -x_0 \|         & < \frac{\epsilon}{|\lambda |} |\lambda | = \epsilon \\
    \| \lambda \langle x,y \rangle_F - \lambda x_0  \| & < \epsilon
    \| <\lambda x, y> _F - \lambda x_0 \|              & < \epsilon
\end{align*}
Where in the last equality we used the linearity of the norm in a finite-dimensional setting.

\subsubsection{Positive semi-definiteness}
This part is trivial as \(\langle x_i,x_i \rangle \) for any \(i \in I\). It follows that
\begin{equation}
    \langle x,x \rangle = \sum_{i \in  I}\langle x_i,x_i \rangle  > 0.
\end{equation}
Then the so-defined inner product satisfies the inner product properties.

\subsubsection{Well definiteness}
Here we show that for every \(x,y \in  H\), \(\langle x,y \rangle \in  \mathbb{K}\).
\begin{align*}
    \langle x,y \rangle & = \sum_{i \in  I}\langle x_i, y_i \rangle       \\
                        & \leq  \sum_{i \in  I}|\langle x_i,y_i \rangle | \\
                        & \leq \sum_{i \in I}\| x_i \|\| y_i \|           \\ &\leq \sum_{i \in  I} \| x_i \|^{2} + \sum_{i \in  I}\| y_i \|^{2} < \infty.
\end{align*}
Where in the first inequality we used the Cauchy-Schwarz inequality.
\subsubsection{Completeness}
We now proceed to prove the completeness of \(H\).
Let \((h_n)_{n \in  \mathbb{N}}\) be a Cauchy sequence, that is, for a certain \(N>0\), then for all \(n,m>N\) \(\| h_n - h_m \| <\epsilon \). Therefore, \(\| h_n(i) - h_m(i) \|<\epsilon  \) is also a Cauchy sequence in \(H_i\), thus it converges to the value say \(h(i)\). Consider then \(\lim_{n \to \infty}h_n(i) = h(i) \) the candidate element for the Cauchy sequence to converge to.

Set \(N_0\) so that for every \(n,m >N_0\) \(\| h_n -h_m\|< \frac{\epsilon^2}{2} \). Thus, simply by the definition of \(h\) we get that for every \(i \in  I\) the following holds true.
\begin{align*}
    \lim_{m \to \infty}\| h_n(i) - h_m(i) \| = \| h_{n}(i)-h(i)  \| \\.
\end{align*}
Let \(G \subset I\) finite. Then
\begin{align*}
    \sum_{i \in  G}\| h_n(i)- h(i) \|^{2} & = \sum_{i \in  G} \lim_{m \to \infty} \| h_n(i) -h_m(i)\|                      \\
                                          & = \lim_{m \to \infty} \sum_{i \in  G} \| h_n(i) -h_m(i)\|                      \\
                                          & < \lim_{m \to \infty} \sum_{i \in  I} \| h_n(i) -h_m(i)\|                      \\
                                          & = \lim_{m \to \infty} \| h_n(i) - h_m(i)\| <\frac{\epsilon^2}{2}  < \epsilon^2
\end{align*}

By the triangle inequality, we prove that \(h \in  H\).
\begin{align*}
    \| h \|^2 & = \sum_{i \in  I} \| h(i) \|^{2}                                                             \\
              & \leq \sum_{ i \in  I} (\| h_n(i)- h(i)  \| +\| h_{n}(i)  \| ) ^{2}                           \\
              & = \sum_{i \in I} \| h_n(i)-h(i) \|^{2} + \| h_n(i) \|^{2}  + \| h_n(i)- h(i) \| \| h_n(i) \| \\
              & \leq  \| h_{n} - h  \| + \| h_n \|  + \| h_n-h \| ^{2}  + \| h_n \|^{2}  < \infty
\end{align*}

\subsection{}
Let \(x_i = \sum_{k} \mathcal{E}_i^k \alpha _i^k \) for every \(x_i \in  H_i\) where the subscript \(k\) is used as second index instead of power.
By definition, \(x  \in  H \implies  \bigcup_{i \in I}^{\infty} x_{i} = \sum_{ i \in I} \sum_{k} \mathcal{E}^k_i \alpha _i^k  \).
This is sufficient to claim that \(\bigcup_{i \in I}\mathcal{E}_i \) is a basis in \(H\).

\subsection{}
Let's start by proving the first implication.

Given that each \(H_i\) is countable, we take a countable basis \(B_i \subset H_i\). By the previous point, \(B\coloneqq \bigcup_{i=1}^{\infty} B_i\) is a basis for \(H\). Given that each \(B_I\) is countable and we take a countable union, it implies that \(B\) is also countable. Given that \(H\) has a countable basis \(B\) it implies that \(H\) is itself countable.

Let's now prove the other implication. For every \(H_i \in  H\) take \(V_i \in  H_i\) open. Thus, \((V_i)_{i \in I}\) is a collection of disjoint open sets and it is uncountable, therefore \(H\) is not second countable, which is equivalent to not being separable as \(H\) is a metrizable space.

\subsection{}
Assume \(\sup_{i \in I}A_i =k<\infty \), and assume that \(\| A \|_{op} = k_i \). Then, for every \(x \in  H \) such that \(\| x \| \leq 1\) the following holds true.
\begin{equation}
    \| Ax \| ^2 = \sum_{i \in  I} \| A_i x_i \| ^{2} \leq \| k_i x_i \|^{2}.
\end{equation}
Then, \(\sup \| Ax \|^2 = \sup \sum_{i \in  I}k_i^{2}x_i^2 \leq k \sum_{i \in  I}x_i < \infty        \) where the supremum is taken over the elements \(x \) with \(\| x\|\leq 1 \).

For the converse (\(\implies \) ), assume \(\exists j \in  I\) such that \(\| A_k \| = \infty  \), i.e \(\forall N >0, \exists  h_n  \) such that \(\| h_n \|\leq 1 \) and \(\| A_j h_n \| >N\).
Take now \(\overline{x} \in  H \) such that \(\overline{x}(j)=h_n\) and \(\overline{x}(i) = 0  \) for all \(i\neq j\). It follows that \(\| x \|\leq 1 \) and that
\begin{equation}
    \| A \|  = \sup \{ \| Ax \| , \| x \|\leq 1  \} \geq \| A \overline{x}  \| = A_j x = \infty
\end{equation}


In order to prove the equality, we note from the first part that
\begin{equation}
    \sup \| Ax \| ^2 \leq  \sup _i \| A_i \| \sum_{i}\| x_i \|^2 = \sup \| A_i \| \| x \|.
\end{equation}

We now prove the other inequality.
\begin{equation}
    \| A_i x_i \|_{H_i}^2 \leq  \sum_{i_I} \| A_i x_i \|^{2} = \sum_{i \in I}\| A x\|^2_{H_i} = \| Ax \|^2 _H \leq  \| A \| ^2.
\end{equation}
Thus \(\sup _i \| Ai \|^2 \leq  \| A \| ^2 <\infty  \). Both inequalities conclude the desired equality.

\section{Exercise 4}
\subsection{}

Let's start noting how the operator \(S\) affects the inner product.

\[
    \langle Sx,y \rangle = \langle (0,x_1  , \dots ),y \rangle = \sum_{i=1}^{\infty} x_i y_{i+1}
\]
We can see that the inner product with \(x\) and the adjoint on \(y\) must be equal  \(\langle x,S^*y \rangle=\sum_{i=1}^{\infty} x_i y_{i+1} \)
which is the left-shift operator
\[
    S^*(x_1,x_2,\dots ) = (x_2,x_3, \dots  )
\]
\subsection{}
Let's compute now the concatenation of \(S\) and its adjoint.
\[
    SS^*(x_1,x_2,\dots )=S(x_2,x_3, \dots  ) = (0,x_2,x_3, \dots  )
\]

Conversely
\[
    S^*S(x_1,x_2,\dots )=S(0,x_1,x_2,x_3, \dots  ) = (x_1,x_2, \dots  ) = x
\]

\subsection{}
And we can extend this to \(S^n\) and \((S^*)^n\)

\[
    S^n (S^*)^nx = (0^n,x_{n+1}, n_{n+2},\dots )
\]
and similarly
\[
    (S^*)^n S^n = x
\]



\end{document}
